\documentclass[11pt]{article}

\usepackage{hyperref}
\usepackage{amsmath}
\usepackage{enumerate}
\usepackage{tikz}
\usepackage{verbatim}
\usetikzlibrary{arrows}
\usepackage{fullpage}
\newcommand{\bigO}{\ensuremath{\mathcal{O}}}

\title{\textbf{Algoritmen en Complexiteit HW 6}}
\author{Jelte Fennema (10183159)\\
		Jaap Koetsier (10440615)\\
        Abe Wiersma (10433120)}
\date{6 maart 2014}

\begin{document}

\maketitle

\begin{enumerate}
    \item
        original:
        $$((\neg x_1 \wedge x_5 \wedge \neg x_3)  \vee (x_2 \wedge x_5)  \vee (x_2 \wedge \neg x_4 \wedge x_1 \wedge x_3)) \wedge ((\neg x_4 \wedge x_1) \vee (\neg x_1 \wedge x_2 \wedge x_5) \vee (\neg x_1 \wedge x_3 \wedge x_1 \wedge \neg x_5))$$
        cnf:
        $$(\neg x_1 \vee \neg x_4) \wedge (x_1 \vee x_5) \wedge x_2 \wedge (x_3 \vee x_5)$$
        3-cnf:
        $$(\neg x_1 \vee \neg x_1 \vee \neg x_4) \wedge (x_1 \vee x_1 \vee x_5) \wedge (x_2 \vee x_2 \vee x_2) \wedge (x_3 \vee x_3 \vee x_5)$$
    \item
        \begin{enumerate}
            \item
            \item
            \item
            \item
        \end{enumerate}
    \item
        \begin{enumerate}
            \item
                \begin{enumerate}[a]
                    \item Als er twee routes gevonden zijn voor beide chauffeurs is eenvoudig te controleren of deze inderdaad korter zijn dan $K_1$ en $K_2$ door de route langs te lopen en de afstanden tussen alle punten van de routes bij elkaar op te tellen.
                    \item We gaan $K_1$ en $K_2$ apart langs. We halen steeds \'{e}\'{e}n kant weg en trekken de desbetreffende waarde van $K$ af. We gaan door totdat er nog slechts \'{e}\'{e}n kant (twee knopen) overblijft. Als de overgebleven kant kleiner is dan de $K$ behorende bij dit pad, en dit geldt voor zowel $K_1$ als $K_2$, hebben we inderdaad een oplossing voor ons probleem.
                    \item Dit probleem is om te vormen in het travelling salesman problem. Als we een route gevonden hebben in de vorm van het TSP kunnen we de route die kleiner is dan $K_{1+2}$ daarna in twee\"{e}n delen zodat $K_1$ en $K_2$ aan de voorwaarden voldoen.
                \end{enumerate}
            \item
                \begin{enumerate}[a]
                    \item Als we een groep van $K$ personen gevonden hebben
                        maken we een lijst van de verenigingen waarvan de
                        personen lid zijn. Als alle verenigingen in de lijst
                        hoogstens één keer voorkomen hebben we een
                        oplossing voor ons probleem.

                    \item We maken een graaf $G$ met als knopen personen en
                        verenigingen. Iedere persoon is met kanten verbonden aan
                        de verenigingen waar deze lid van is. Subverenigingen en
                        verenigingen zijn hetzelfde verbonden.
                        Men zoekt een 2 leden waar een pad tussen is en haalt
                        het eerste lid uit de set. Het probleem wordt dan dus
                        gereduceerd van $(G, K)$ tot $(G - \{l\}, K)$.

                    \item
                        Het VERTEX COVER probleem is te reduceren tot een UNIONS
                        probleem door alle knopen als mensen te zien en alle
                        kanten als verenigingen. De mensen zitten dan in de
                        verenigingen van de edges waar ze niet aan vast liggen.

                \end{enumerate}
            \item
                \begin{enumerate}[a]
                    \item
                        Het certificaat is een set $V'$. Dit is dan te
                        controleren door alle knopen in $V'$ uit de set te
                        halen en dan te kijken of er nog cykels in de $G - V'$
                        zitten.

                    \item
                        Vind een cykel $C$ in $G$. Het probleem is dan te
                        reduceren voor elke knoop $v_{1...n}$ in $C$ dit $(G -
                        \{v\}, K - 1)$ te testen, als daar 1 van waar is dan is
                        het originele probleem ook op te lossen. Het
                        uiteindelijke gereduceerde probleem is dus $(G -
                        \{v_1\}, K - 1) \lor ... \lor (G - \{v_n\}, K - 1)$

                    \item
                        Het VERTEX COVER probleem is te reduceren tot een
                        FEEDBACK VERTEX SET probleem door $G$ uit te bereiden
                        met een knoop voor elke kant. Dan moeten er kanten
                        worden getrokken van de knopen voor de kanten naar de
                        knopen waar die kant mee verbonden is.

                \end{enumerate}
            \item
                \begin{enumerate}[a]
                    \item
                        Als er een oplossing gegeven wordt kun je door de
                        variabelen in te vullen de oplossing binnen polynomiale
                        tijd verifi\"{e}ren.

                    \item
                        We nemen de volledige vergelijkingen. We vervangen
                        \'{e}\'{e}n variabele voor $0$ en $1$. Als één van deze
                        nieuwe set vergelijkingen is op te lossen door 0 of 1 in
                        te vullen voor de overgebleven variabelen is het
                        originele probleem ook op te lossen. We reduceren
                        de vergelijkingen dus met één variabele.

                    \item
                        SATISFIABILITY is vrij makkelijk te reduceren naar dit
                        probleem. De set lineare vergelijkingen bestaat alleen
                        uit de formule $F=1$.

                \end{enumerate}

        \end{enumerate}
\end{enumerate}

\end{document}
