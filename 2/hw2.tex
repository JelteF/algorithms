\documentclass[11pt]{article}

\usepackage{hyperref}
\usepackage{amsmath}
\usepackage{enumerate}
\usepackage[utf8]{inputenc}

\newcommand{\bigO}{\ensuremath{\mathcal{O}}}

\title{\textbf{Algoritmen en Complexiteit HW 2}}
\author{Jelte Fennema (10183159)\\
		Jaap Koetsier (10440615)\\
        Abe Wiersma (xxxxxxxxxx)}
\date{14 februari 2014}

\begin{document}

\maketitle

\begin{enumerate}
    \item opdraht 1
        \begin{enumerate}
            \item
            \item
            \item
            \item
        \end{enumerate}
    \item
        Allereerst berekent men het kortste pad tussen $s$ en $t$, dit wordt
        gedaan met Dijkstra's algoritme. Daarna haalt men de eerste kant uit dat
        pad uit de graaf. Men berekent opnieuw het kortste pad tussen $s$ en
        $t$. De lengte van dit pad wordt bewaard. Dit herhaalt men voor alle
        kanten uit het kortste pad tussen $s$ en $t$ (Dijkstra's). De kant waarbij het
        langste pad is gevonden is de kant die weg gehaald moet worden.

        De complexiteit van Dijkstra's algoritme is $$\bigO(|E| + |V|\log|V|).$$
        Dit wordt initiëel gedaan. Daarna wordt nog $x$ aantal keer Dijkstra's
        algoritme uitgevoerd dus weer met complexiteit
        $$\bigO(|E| + |V|\log|V|).$$
        Het initiële kortste pad kan maximaal $|V|-1$ zijn dus de complexiteit
        is $$\bigO(|V||E| + |V|^2\log|V|)$$
    \item
        \begin{enumerate}
            \item
                De 3 voorwaardes voor een greedy algoritme zijn:

                \begin{enumerate}[\bfseries 1.]

                    \item Een optimale oplossing wordt stap voor stap opgebouwd.

                    \item Een éénmaal genomen stap wordt nooit ongedaan gemaakt.

                    \item In iedere stap wordt een zo groot of zo goed mogelijk
                        tussenresultaat behaald.

                \end{enumerate}

                Deze worden allemaal behaald op deze manier:

                \begin{enumerate}[\bfseries 1.]

                    \item Er wordt stap voor stap gezocht gezocht naar een cykel
                        tot die gevonden is en dan wordt de volgende gezocht.

                    \item Er worden nooit kanten toegevoegd.

                    \item Je haalt altijd de zwaarste kant uit de cykel,
                        waardoor je de lichtste kanten over houdt in je
                        (tussenvorm van de) boom.

                \end{enumerate}

            \item De complexiteit is hetzelfde als die van breadth first search,
                dus $\bigO(|V|)$
        \end{enumerate}


    \item

\end{enumerate}

\end{document}
