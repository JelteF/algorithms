\documentclass[11pt]{article}

\usepackage{hyperref}
\usepackage{amsmath}
\usepackage{enumerate}
\usepackage{tikz}
\usepackage{verbatim}
\usetikzlibrary{arrows}
\newcommand{\bigO}{\ensuremath{\mathcal{O}}}

\title{\textbf{Algoritmen en Complexiteit HW 4}}
\author{Jelte Fennema (10183159)\\
		Jaap Koetsier (10440615)\\
        Abe Wiersma (10433120)}
\date{6 maart 2014}

\begin{document}

\maketitle

\begin{enumerate}
    \item Zie bijlage hw5.py
    \item
    Men neemt een taal die de lengte van de invoer neemt, dat kwadrateert en dan voor de uitkomst daarvan nullen schrijft op de band. Dit probleem heeft per definitie een geheugengebruik van \bigO$(n^2)$ 
    \item Een reduceerfunctie is per definitie een eindige functie met een bepaalde output. Dit probleem is niet te vatten in een of andere reduceerfunctie waarbij het algoritme tot een halt komt. De Turingmachine zal met deze taal oneindig doorgaan. We hebben immers een oneindige band met blanco cellen aan de uiteinden. Wanneer we een blanco cel tegenkomen schrijven we een 0 en gaan we verder. Dit is niet te vatten in een of andere functie die ons probleem reduceert, dus is de taal onbeslisbaar.
    \item
        \begin{enumerate}
            \item Het probleem is op te lossen door door middel van 'exhaustive search' de optimale bezetting van klanten in snackbars te vinden. In het optimale geval wil je natuurlijk elke snackbar op maximale capaciteit aan het werk zien. Stel dat met de eerste klant in de eerste snackbar de snackbar al te vol zit om er nog een tweede klant bij te stoppen, dan kom je hier pas achter nadat je alle andere klanten geprobeerd hebt hierbij te zetten. Ook een greedy algoritme (eerst de grootste eters in een hokje stoppen) hoeft de oplossing niet te vinden, terwijl deze er misschien wel is. \\Voorbeeld: Stel je hebt op een gegeven moment nog drie snackbars $a, b$ en $c$ met respectievelijk $3, 5$ en $4$ vrije frikandellen per uur. We hebben Jan $j$, Kees $k$, Mees $m$ en Nico $n$ nog over die resp. $4, 3, 3$ en $2$ frikandellen per uur eten. Zouden we Jan greedy in snackbar $b$ zetten, dan komen Kees en Mees en $a$ en $c$ terecht, waarna cafetaria $b$ en $c$ beide nog ruimte overhebben voor één frikandel per uur, maar Nico heeft honger voor $2$. Eenvoudig te zien is dat het probleem wel opgelost had kunnen worden, maar een greedy algoritme geeft geen oplossing.\\Met andere woorden: je moet blijven combineren, totdat alle klanten zo gepositioneerd zijn dat ze allemaal te eten krijgen. Wanneer de juiste bezetting eenmaal gevonden is is de oplossing echter eenvoudig te verifi\"{e}ren. Dit kan je namelijk doen door de honger van alle klanten bij een snackbar bij elkaar op te tellen en te kijken of dit niet meer is dan de capaciteit van de snackbar. Dit moet je dan vooor alle snackbars doen. 

            \item De oplossing van dit probleem vind je alleen door zoveel mogelijke onafhankelijke taken in een batch te stoppen. Het probleem is dat het toevoegen van een onafhankelijke taak $a$ in een batch $1$ ervoor kan zorgen dat er voor een andere taak $b$ weer een nieuwe batch $5$ gestart moet worden, omdat alle overige batches $1$ t/m $4$ niet ofafhankelijk van $b$ zijn. Mogelijk was dit niet nodig geweest als $b$ ofafhankelijk geweest was van alle taken in batch $1$ voordat taak $a$ hieraan toegevoegd werd, en $a$ onafhankelijk geweest was van de taken in de batches $2$ t/m $4$.\\ Een oplossing vind je door te blijven combineren.\\ Weer geldt: als een oplossing eenmaal gevonden is kunnen we heel eenvoudig verifi\"{e}ren of we maximaal $k$ batches hebben en of de taken in elke batch onafhankelijk van elkaar zijn, en we dus een juiste oplossing gevonden hebben. 

            \item Om te weten wat de maximale stroom door de graaf is zul je alle kanten moeten doorlopen om te zien of je alle paden (stroomrichtingen) benut hebt om de maximale stroom te bereiken. Eerder weet je niet of je de maximale stroom bereikt hebt. Wanneer je eenmaal weet wat het antwoord is, is dit eenvoudig te verifi\"{e}ren met (bijvoorbeeld) het Ford-Fulkerson algoritme.

        \end{enumerate}

\end{enumerate}

\end{document}
