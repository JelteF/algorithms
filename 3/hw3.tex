\documentclass[11pt]{article}

\usepackage{hyperref}
\usepackage{amsmath}
\usepackage{enumerate}
\usepackage[utf8]{inputenc}

\newcommand{\bigO}{\ensuremath{\mathcal{O}}}

\title{\textbf{Algoritmen en Complexiteit HW 3}}
\author{Jelte Fennema (10183159)\\
		Jaap Koetsier (10440615)\\
        Abe Wiersma (10433120)}
\date{14 februari 2014}

\begin{document}

\maketitle

\begin{enumerate}
    \item

    \item
        Voor elke doos kijken we wat de maximale hoogte van de stapel is als
        deze bovenop staat. Dit doen we voor alle stapelhoogtes. Dit wordt
        gedaan door te kijken op welke van de stapels uit de vorige stapelhoogte
        hij geplaatst kan worden. Hij kan op een stapel geplaatst worden waar de
        doos nog niet in gebruikt is en de breedte en diepte van de bovenste
        doos niet groter is als zijn eigen breedte en diepte. Van deze gevonden
        stapels wordt de hoogste gekozen.

        Past hij niet op een stapel, dan is de hoogte van die stapel 0. Past hij
        wel op een stapel dan wordt zijn hoogte bij de hoogte van die stapel
        opgeteld.

        Aan het einde wordt gekeken welke van de stapels het hoogste is.

        De stapelhoogte is maximaal $n$, waarbij voor elke stapelhoogte voor
        alle $n$ dozen wordt gekeken op welke van de $n - 1$ lagere stapels hij
        past. Dit algorithme heeft dus complexiteit \bigO$(n^3)$.

    \item
        \begin{enumerate}
            \item
                Een voorbeeld is een balk van lengte 8 waar bij de sneden op
                posites 3, 4 en 5 gemaakt moeten worden.

                De door het algoritme gekozen volgorde zou zijn: $4, 3, 5$ of
                $4, 5, 3$. Dit komt uit op kosten van $8+4+4=16$. Terwijl het
                goedkoper zou zijn om in volgorde $3, 5, 4$ te zagen. Dit geeft
                de kosten $8+5+2=15$

            \item
                Voor 1 tot uiteindelijk aantal snedes kijken we wat de
                laagst mogelijke toevoeging aan de prijs. Dit doen we door te
                kijken naar de vorige kolom van de matrix naar alle rijen
                behalve zijn eigen. Hieruit kiest het algoritme de
                deelverzameling waarbij zijn snede de minste toevoeging geeft
                aan het totaal.
        \end{enumerate}

    \item
        \begin{enumerate}
            \item
            \item
        \end{enumerate}

\end{enumerate}

\end{document}
